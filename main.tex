% mnras_template.tex 
%
% LaTeX template for creating an MNRAS paper
%
% v3.0 released 14 May 2015
% (version numbers match those of mnras.cls)
%
% Copyright (C) Royal Astronomical Society 2015
% Authors:
% Keith T. Smith (Royal Astronomical Society)

% Change log
%
% v3.0 May 2015
%    Renamed to match the new package name
%    Version number matches mnras.cls
%    A few minor tweaks to wording
% v1.0 September 2013
%    Beta testing only - never publicly released
%    First version: a simple (ish) template for creating an MNRAS paper

%%%%%%%%%%%%%%%%%%%%%%%%%%%%%%%%%%%%%%%%%%%%%%%%%%
% Basic setup. Most papers should leave these options alone.
\documentclass[fleqn,usenatbib]{mnras}

% MNRAS is set in Times font. If you don't have this installed (most LaTeX
% installations will be fine) or prefer the old Computer Modern fonts, comment
% out the following line
\usepackage{newtxtext,newtxmath}
% Depending on your LaTeX fonts installation, you might get better results with one of these:
%\usepackage{mathptmx}
%\usepackage{txfonts}

% Use vector fonts, so it zooms properly in on-screen viewing software
% Don't change these lines unless you know what you are doing
\usepackage[T1]{fontenc}
\usepackage{ae,aecompl}


%%%%% AUTHORS - PLACE YOUR OWN PACKAGES HERE %%%%%

% Only include extra packages if you really need them. Common packages are:
\usepackage{graphicx}	% Including figure files
\usepackage{amsmath}	% Advanced maths commands
\usepackage{amssymb}	% Extra maths symbols
\usepackage{lscape}
\usepackage{rotating}
%%%%%%%%%%%%%%%%%%%%%%%%%%%%%%%%%%%%%%%%%%%%%%%%%%

%%%%% AUTHORS - PLACE YOUR OWN COMMANDS HERE %%%%%

% Please keep new commands to a minimum, and use \newcommand not \def to avoid
% overwriting existing commands. Example:
%\newcommand{\pcm}{\,cm$^{-2}$}	% per cm-squared

%%%%%%%%%%%%%%%%%%%%%%%%%%%%%%%%%%%%%%%%%%%%%%%%%%

%%%%%%%%%%%%%%%%%%% TITLE PAGE %%%%%%%%%%%%%%%%%%%

% Title of the paper, and the short title which is used in the headers.
% Keep the title short and informative.
\title[Radiative Transfer of Star-Disk Interactions]{Radiative Transfer Simulations of Star-Disk Interactions in T Tauri Systems}

% The list of authors, and the short list which is used in the headers.
% If you need two or more lines of authors, add an extra line using \newauthor
\author[T. J. G. Wilson et al.]{
T. J. G. Wilson,$^{1}$\thanks{E-mail: tjgw201@exeter.ac.uk}
S. Matt,$^{1}$
T. Harries$^{1}$
\\
% List of institutions
$^{1}$College of Engineering, Mathematics and Physical Sciences, Physics Building, Streatham campus,University of Exeter , Exeter EX4 4QL, UK\\
}

% Enter the current year, for the copyright statements etc.
\pubyear{2019}

% Don't change these lines
\begin{document}
\label{firstpage}
\pagerange{\pageref{firstpage}--\pageref{lastpage}}
\maketitle

% Abstract of the paper
\begin{abstract}
\noindent
\end{abstract}

%%%%%%%%%%%%%%%%%%%%%%%%%%%%%%%%%%%%%%%%%%%%%%%%%%

%%%%%%%%%%%%%%%%% BODY OF PAPER %%%%%%%%%%%%%%%%%%

\section{Introduction}
Understanding the mechanism by which collapsing interstellar clouds loose
angular momentum (hereafter AM) allowing them to form pre-main sequence stars
is critical to furthering the study of star and planet formation. Without the
loss of AM the difference in scale of the natal cloud to the protostar would
result in a star that was rotating faster than its break up angular
velocity~\citep{Hartmann:2016gu}. 

Progenitors to Sun like objects are T Tauri stars. These are low mass
($\lesssim 2 \textrm{M}_\odot$) pre-main sequence stars which show evidence of 
accretion from a surrounding accretion disk~\citep{1998ApJ...495..385H}. These
young stars ($\lesssim3\times10^{6}~\textrm{yrs}$) have strong
H$\alpha$ emission and excess UV and infrared continuum
emission~\citep{2005MNRAS.358..671K}.
The line profiles exhibit both blue shifted absorption and inverse P Cygni
features, thought to be evidence of outflows and inflows respectively.
Observed accretion rates ($10^{-9}~\textrm{to}~10^{-7}~{\rm M_{\odot} yr^{-1}}$) are expected to be great enough to transfer enough AM to the star to spin it up to well above break up angular velocity $\Omega_B = \sqrt{\frac{GM_{*}}{R_{*}^3}}$~\citep{2005ApJ...632L.135M}. However a relevant fraction of these stars, once they become visible are rotating well below their rotation limit~\citep{1993A&A...272..176B}, and during the few million years of the accretion phase their rotation remains fairly constant. In this phase of their evolution these protostars are still accreting and contracting at rates that should spin them up to break velocity within $\sim10^{6}~{\rm yrs}$~\citep{2009A&A...508.1117Z}. There must be an efficient mechanism to spin down the rotation of the T Tauri stars. 

It is widely accepted that T Tauri stars have
accretion disks. Infrared excess and accretion
signatures point to their presence. In recent years
interferometric observation have confirmed the
presence of disks around young stellar objects for
example the slightly larger Herbig star system shown
by \cite{2012ApJ...752...11K} to have a circumstellar
disk. The accretion mechanism of T Tauri stars is
thought to be dominated by the strong magnetic fields
present, $\sim 10^{3}~\textrm{G}$~\citep{2012MNRAS.426.2901K}.
\citet{1991ApJ...370L..39K} proposed that the accretion mechanism is
similar to that of neutrons stars proposed
by~\citet{1977ApJ...217..578G}. The magnetic field truncates the
accretion radius at some radii $R_T$. where the magnetic and material
stresses are of the same order; $B^2/8\pi = p + \rho v^2$, where $B$,
$p$, $\rho$ and $v$ are the magnetic field, pressure, density and
velocity respectively~\citep{Romanova:2002hc}. Material is lifted from
the disk and free falls to the stellar surface where it forms a shock region as the in falling matter exceeds the sound speed of the plasma. The large energy release in these shock zones could be responsible for the high UV excess observed~\citep{2009A&A...508.1117Z}. The accreting matter carries AM to the star and acts to `spin-up' the star. For a given accretion rate $\dot{M}$ the `spin-up' rate can be quantified as a torque~\citep{2005ApJ...632L.135M}
\begin{equation}
    \tau_{acc} = \dot{M}_{acc}\sqrt{GM_{\ast}R_T}
    \label{eq:accretion_torque}
\end{equation}
here $M_{\ast}$ is the stellar mass. 

The means by which these T Tauri stars are being braked is still debated~\citep{Hartmann:2016gu}. The possible mechanisms can be split broadly into two types; star-disk interactions and outflows. 

The star-disk interactions rely on Magnetohydrodynamic (MHD) mechanism to remove angular momentum from the star. Magnetic fields threading through the disk and star will provide a negative (or `spin-down') torque if they pass through the accretion disk at radii beyond the corotation radius. The poloidal field lines becomes twisting as the conductive disk opposes their toroidal passage~\citep{Uzdensky:2002dg} resulting in a `spin-down' torque on the star. 
However it not clear whether a `spin-down' torque great enough to match the AM transferred by accretion can be achieved by this magnetic breaking.~\citet{2009A&A...508.1117Z} concluded that for their numerical example the spin down torque from the magnetospheric star-disk ``locking'' could only account for $10\%$ of the spin up torque from accretion.

















\section{Conclusions}

\section*{Acknowledgements}


%%%%%%%%%%%%%%%%%%%%%%%%%%%%%%%%%%%%%%%%%%%%%%%%%%

%%%%%%%%%%%%%%%%%%%% REFERENCES %%%%%%%%%%%%%%%%%%

% The best way to enter references is to use BibTeX:

\bibliographystyle{mnras}
\bibliography{library} % if your bibtex file is called example.bib

%%%%%%%%%%%%%%%%%%%%%%%%%%%%%%%%%%%%%%%%%%%%%%%%%%

%%%%%%%%%%%%%%%%% APPENDICES %%%%%%%%%%%%%%%%%%%%%

\appendix

\section{Some extra material}


%%%%%%%%%%%%%%%%%%%%%%%%%%%%%%%%%%%%%%%%%%%%%%%%%%

% Don't change these lines
\bsp	% typesetting comment
\label{lastpage}
\end{document}

% End of mnras_template.tex